\section{Korištene tehnologije}

\subsection{Laravel}
Laravel je PHP (engl. \textit{PHP: Hypertext Preprocessor}) razvojni okvir (engl. \textit{framework}) namijenjen razvoju web aplikacija zasnivanih na MVC (\textit{"Model-View-Controller"}) arhitekturi. Njegove značajke su da je slobodan, besplatan te otvorenog k\^oda, stoga je cijeli njegov izvorni k\^od dostupan na platformi GitHub~\cite{laravelGitHub} gdje svatko može dati svoj doprinos pa se naziva i razvojnim okvirom zajednice (engl. \textit{community framework}). Tvorac Laravela je Taylor Otwell, koji je 2011. godine razvio početnu verziju Laravela pokušavajući poboljšati tada popularan razvojni okvir CodeIgniter.

Trenutno je, uz Symfony, Laravel najpopularniji PHP razvojni okvir zahvaljujući svojoj jednostavnoj i izražajnoj sintaksi, detaljnoj dokumentaciji i obilnoj količini (video)vodiča, jasnoj strukturi, iznimnoj skalabilnosti te bogatom Laravel ekostustavu s dostupnom ogromnom bibliotekom pomno održavanih paketa.

Pruža elegantna gotova \textit{"out-of-the-box"} rješenja za česte značajke potrebne web aplikacijama: jednostavno i brzo usmjerivanje zahtjeva, moćno ubrizgavanje ovisnosti, pohranjivanje sesija i predmemorije, autentikaciju, autorizaciju, interakciju s podatcima u bazi, modelima, migracijama i relacijama, validaciju, slanje e-pošte i notifikacija, pohranu datoteka, red čekanja i pozadinsku obradu poslova, zakazivanje zadataka, događaje itd~\cite{laravel}.

Dolazi s \textbf{Artisan} sučeljem naredbenog retka (engl. \textit{command line interface}) koje pruža mnoge korisne naredbe koje pomažu pri razvoju aplikacije. Artisan naredbe su u formi \texttt{php artisan <command>}.\\ 
Instalacijom Laravela također se dobiva i \textbf{Tinker} (REPL - \textit{"Read—Eval—Print—Loop"}) interaktivna ljuska koja omogućava testne interakcije s modelima, poslovima, događajima itd. Za početak rada u Tinker razvojnom okruženju pokreće se Artisan naredba \texttt{php artisan tinker}~\cite{artisanConsole}.

Laravel i njegovi ostali paketi prve strane (engl. \textit{first-party}) prate semantičko verzioniranje, pri čemu se \textit{major} verzija razvojnog okvira Laravel sada izdaje svake godine, dok su \textit{minor} i \textit{patch} izdanja češća. Sva izdanja razvojnog okvira Laravel primaju ispravke pogrešaka do 18 mjeseci od izdavanja, a sigurnosne ispravke do 2 godine od izdavanja.
Trenutna verzija razvojnog okvira Laravel je 10.x te zahtijeva minimalnu PHP verziju 8.1~\cite{releaseNotes}. 
 
\subsection{Docker spremnici kao razvojno okruženje}

\subsubsection{NGINX}

\subsubsection{PHP}

\subsubsection{MySQL}

\subsubsection{phpMyAdmin}

\subsection{Upravitelji paketima i ovisnostima}

\subsubsection{NPM}

\subsubsection{Composer}

\subsection{Razvojna okruženja za testiranje}

\subsubsection{PHPUnit}

\subsubsection{Pest}
