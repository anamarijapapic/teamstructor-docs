\section{Korištene tehnologije}

\subsection{Laravel}
Laravel je PHP (engl. \textit{PHP: Hypertext Preprocessor}) razvojni okvir (engl. \textit{framework}) namijenjen razvoju web aplikacija zasnivanih na MVC (\textit{"Model-View-Controller"}) arhitekturi. Njegove značajke su da je slobodan, besplatan te otvorenog k\^oda, stoga je cijeli njegov izvorni k\^od dostupan na platformi GitHub~\cite{laravelGitHub} gdje svatko može dati svoj doprinos pa se naziva i razvojnim okvirom zajednice (engl. \textit{community framework}). Tvorac Laravela je Taylor Otwell, koji je 2011. godine razvio početnu verziju Laravela pokušavajući poboljšati tada popularan razvojni okvir CodeIgniter.

Trenutno je, uz Symfony, Laravel najpopularniji PHP razvojni okvir zahvaljujući svojoj jednostavnoj i izražajnoj sintaksi, detaljnoj dokumentaciji i obilnoj količini (video)vodiča, jasnoj strukturi, iznimnoj skalabilnosti te bogatom Laravel ekostustavu s dostupnom ogromnom bibliotekom pomno održavanih paketa.

Pruža elegantna gotova \textit{"out-of-the-box"} rješenja za česte značajke potrebne web aplikacijama: jednostavno i brzo usmjerivanje zahtjeva, moćno ubrizgavanje ovisnosti, pohranjivanje sesija i predmemorije, autentikaciju, autorizaciju, interakciju s podatcima u bazi, modelima, migracijama i relacijama, validaciju, slanje e-pošte i notifikacija, pohranu datoteka, red čekanja i pozadinsku obradu poslova, zakazivanje zadataka, događaje itd~\cite{laravel}.

Dolazi s \textbf{Artisan} sučeljem naredbenog retka (engl. \textit{command line interface}) koje pruža mnoge korisne naredbe koje pomažu pri razvoju aplikacije. Artisan naredbe su u formi \texttt{php artisan <command>}.\\ 
Instalacijom Laravela također se dobije i \textbf{Tinker} (REPL - \textit{"Read—Eval—Print—Loop"}) interaktivna ljuska koja omogućava testne interakcije s modelima, poslovima, događajima itd. Za početak rada u Tinker razvojnom okruženju pokreće se Artisan naredba \texttt{php artisan tinker}~\cite{artisanConsole}.

Laravel i njegovi ostali paketi prve strane (engl. \textit{first-party}) prate semantičko verzioniranje, pri čemu se \textit{major} verzija razvojnog okvira Laravel sada izdaje svake godine, dok su \textit{minor} i \textit{patch} izdanja češća. Sva izdanja razvojnog okvira Laravel primaju ispravke pogrešaka do 18 mjeseci od izdavanja, a sigurnosne ispravke do 2 godine od izdavanja.
Trenutna verzija razvojnog okvira Laravel je 10.x te zahtijeva minimalnu PHP verziju 8.1~\cite{releaseNotes}. 
 
\subsection{Docker spremnici kao razvojno okruženje}
Kako bi se web aplikacija jednako uspješno izvodila u različitim okruženjima - na različitim operativnim sustavima i arhitekturama računala, korištena je tehnologija Docker \textbf{spremnika} (engl. \textit{containers}) te je aplikacija kontejnerizirana.

Svaki stroj koji ima instaliran Docker Engine može pokretati i stvarati Docker spremnike konstruirane iz Docker \textbf{slika} (engl. \textit{image}). Za izgraditi sliku spremnika Docker koristi \texttt{Dockerfile} datoteku u kojoj se nalazi skripta s uputama za kreiranje iste. Za inicijalizaciju i izvođenje aplikacija koje se sastoje od više spremnika potreban je alat \textbf{Docker Compose} te su u tom slučaju aplikacijski servisi konfigurirani u \texttt{docker-compose.yml} datoteci. Instalacijom Docker Desktop aplikacije na računalo dobiva se i Docker Engine i Docker Compose V2 te dodatni alati. Upisivanjem naredbe \texttt{docker compose up} u terminal iz direktorija projekta pokreće se aplikacija i svi njeni servisi, a naredbom \texttt{docker compose down} zaustavljaju se svi pokrenuti servisi te se brišu spremnici~\cite{dockerCompose}. Unutar \texttt{docker-compose.yml} datoteke kreirani su imenovani \textbf{volumeni} (engl. \textit{volumes}) koji služe za očuvanje podataka koje generiraju i koriste Docker spremnici~\cite{dockerVolumes}.

U nastavku će kratko biti opisane tehnologije koje su nužne za rad aplikacije, a definirane su kao servisi u \texttt{docker-compose.yml} datoteci.

\subsubsection{Nginx}
Kao lokalni web poslužitelj (engl. \textit{server}) korišten je Nginx [engine x]. Kao početna točka za konfiguraciju web poslužitelja koristi se datoteka \texttt{default.conf}~\cite{nginxDeployment}, u čijem se sadržaju unutar \texttt{server} direktive konfigurira virtualni poslužitelj~\cite{nginxServer}.

\subsubsection{PHP}
PHP (rekurzivni akronim za  \textit{PHP: Hypertext Preprocessor}, a prije je kratica označavala \textit{Personal Home Page}) popularni je besplatni skriptni jezik otvorenog k\^oda (izvorni k\^od je dostupan na platformi GitHub~\cite{phpGitHub}) koji se izvršava na poslužitelju (engl. \textit{server-side scripting}). 1994. godine razvio ga je  Rasmus Lerdorf, a sintaksa mu je bazirana na C, Java i Perl programskim jezicima. PHP je jezik opće namjene, ali je posebno prikladan za razvijanje web aplikacija~\cite{Brekalo}.

PHP se koristi u najpopularnijem sustavu za upravljanje
sadržajem (engl. \textit{CMS - Content Management System}) – WordPressu, a postoji i nekoliko PHP razvojnih okvira: Laravel, Symfony, CodeIgniter, Zend, Yii 2, CakePHP, Fuel PHP, FATFree, Aura i dr.~\cite{Brekalo}

Trenutna verzija PHP jezika je 8.2, a trenutno je aktivno podržana i verzija 8.1. Svaka grana izdanja (engl. \textit{release branch}) PHP-a u potpunosti je podržana dvije godine od svog prvog stabilnog izdanja i tijekom tog razdoblja za nju se objavljuju ispravci pogrešaka i sigurnosnih problema. Nakon tog razdoblja aktivne podrške, grana izdanja dobiva još jednu godinu podrške samo za kritične sigurnosne ispravke i to po potrebi. Nakon što isteknu tri godine podrške, grana dolazi do kraja života (engl. \textit{end of life}) i više nije podržana~\cite{phpVersions}.

\subsubsection{MySQL}
MySQL je besplatni program otvorenog k\^oda za upravljanje relacijskim bazama podataka (engl. \textit{RDBMS - Relational Database Management System}). Pokrata "SQL" u imenu stoji za \textit{“Structured Query Language”} - najčešći standardizirani jezik korišten za pristup bazama podataka. Najčešća alternativa MySQL-u su također besplatni programi otvorenog k\^oda MariaDB i PostgreSQL. Kod MySQL-a osnovni stroj baze podataka i ujedno i najčešće korišteni je InnoDB koji koristi transakcijski mehanizam~\cite{Brekalo}.

\subsubsection{phpMyAdmin}
phpMyAdmin je besplatni softverski alat otvorenog k\^oda napisan u PHP-u koji podržava širok spektar operacija unutar MySQL i MariaDB relacijskih baza podataka. Radi se o intuitivnom grafičkom korisničkom sučelju (engl. \textit{GUI - Graphical User Interface}) pa akcije mogu biti izvršene unutar korisničkog sučelja izravno u web pregledniku~\cite{Brekalo}.

Može mu se pristupiti unosom adrese \url{http://localhost:8080} u web preglednik, pri čemu će se pojaviti phpMyAdmin "Welcome to phpMyAdmin" stranica koja traži unos korisničkog imena i loznike.

\subsection{Upravitelji paketima i ovisnostima}

\subsubsection{npm}
npm (Node package manager) je upravitelj paketa za Node.js nastao 2009. godine kao projekt otvorenog k\^oda koji bi JavaScript programerima pomogao da jednostavno dijele zapakirane module k\^oda.

npm je klijent naredbenog retka koji programerima omogućuje instaliranje i objavljivanje tih paketa.

\textit{Frontend} paketi o kojima aplikacija ovisi zapisani su u \texttt{package.json} datoteku, a te iste ovisnosti (engl. \textit{dependencies}) instaliraju se pozivom naredbe \texttt{npm install} iz terminala. Pokretanjem te naredbe kreira se \texttt{node\_modules} direktorij koji u sebi sadrži poddirektorije - instalirane module tj. biblioteke potrebne za \textit{frontend} dio aplikacije~\cite{npm}.

\subsubsection{Composer}
Composer je alat za upravljanje ovisnostima u PHP-u te je osnova modernog PHP razvoja. Omogućava deklariranje "paketa" tj. biblioteka o kojima projekt ovisi te će njima upravljati - voditi brigu o instalaciji, ažuriranju i uklanjanju istih na bazi projekta tako da će ih preuzeti unutar projekta u \texttt{vendor} direktorij. Unutar \texttt{composer.json}
datoteke zapisane su željene ovisnosti projekta. Unutar \texttt{composer.lock} datoteke zapisani su svi instalirani paketi i njihove točne verzije, tako da se projekt "zaključava" na te konkretne verzije paketa~\cite{composerIntro}.

Određeni paket može se zatražiti naredbom \texttt{composer require} gdje se navodi naziv isporučitelja paketa, naziv paketa te ograničenje verzije paketa. Naredbom \texttt{composer install} instaliraju se sve Composer ovisnosti aplikacije, naredbom \texttt{composer update} ažuriraju se paketi, a naredbom \texttt{composer remove} i navođenjem naziva isporučitelja paketa i naziva paketa može se ukloniti paket iz liste ovisnosti~\cite{composerUsage}.

Minimalna Composer verzija koju Laravel 10.x zahtijeva je 2.2.0.

\subsection{Razvojna okruženja za testiranje}
\textbf{PHPUnit} Sebastiana Bergmanna najpopularnije je PHP razvojno okruženje za testiranje. U Laravelu je podrška za testiranje s PHPUnitom zadano uključena te je \texttt{phpunit.xml} datoteka unaprijed postavljena. PHPUnit 10 je trenutna stabilna verzija.

\textbf{Pest} je razvojno okruženje za testiranje izgrađeno na PHPUnitu, ali s uključenim novim dodatnim značajkama. Podržava i pokretanje testova pisanih za PHPUnit. Trenutna verzija Pesta je 2.0.