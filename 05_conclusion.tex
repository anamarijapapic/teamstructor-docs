\section{Zaključak}
U ovom završnom radu predstavljen je način na koji je razvijen praktični dio završnog rada - Laravel aplikacija \textit{Teamstructor}. Navedene su tehnologije koje su korištene te je detaljno opisan način implementacije pojedinih dijelova aplikacije što je potkrijepljeno popratnim ispisima k\^oda i slikama.

Ova aplikacija omogućava korisnicima da formiraju timove u kojima mogu surađivati na različitim timskim projektima na način da mogu pregledavati projektne resurse - učitane zajednički dijeljene datoteke te imaju prostor za raspravu - mogu objavljivati i komentirati obavijesti i ažuriranja relevantna za pojedini projekt. 

U ovoj aplikaciji realizirani su svi inicijalno postavljeni ciljevi, a zahvaljujući izuzetnoj skalabilnosti odabranog razvojnog okvira po potrebi bi se brzo i lako mogle dodati nove proširene značajke. Imajući u vidu da se radi o aplikaciji koja služi kao platforma za timski rad, moguće proširenje bilo bi integracija s nekom popularnom platformom treće strane kao što je npr. Slack ili s obzirom da je jedna od zadaća aplikacije i pohrana projektnih resursa možda bi se moglo iste i verzionirati.  

Zahvaljujući početnom kompletu Laravel Jetstream pri razvijanju aplikacije ušteđen je dio vremena jer pruža i \textit{scaffolding} za potreban autentikacijski sustav i \textit{frontend scaffolding} Livewire + Blade (uz korištenje Tailwind CSS-a) koji je lako prilagoditi potrebama aplikacije i ujedno korisniku ponuditi dinamičko sučelje moderna i atraktivna izgleda. 

Pohrana datoteka tj. projektnih resursa implementirana je pomoću Spatie Laravel Media Library paketa te se datoteke pohranjuju na AWS S3 kompatibilan servis za pohranu MinIO.

Aplikacija je lokalizirana tj. pruža korisničko sučelje i na početno zadanom engleskom jeziku, ali i na hrvatskom jeziku te je trenutni jezik aplikacije lako promijeniti koristeći se padajućim izbornikom za promjenu jezika u navigacijskoj traci ili unosom URL-a s vrijednošću parametra \texttt{locale} (\texttt{en} ili \texttt{hr}) u web preglednik .

Također, za pojedine dijelove aplikacije pisani su PHPUnit ili Pest \textit{feature} testovi te se kroz cijeli \textit{development} proces nastojalo što je više moguće pratiti preporučene prakse i držati do kvalitete k\^oda.
