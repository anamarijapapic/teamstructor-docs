\section{Uvod}
U današnjem dinamičnom poslovnom svijetu, gotovo je nazamisliv posao koji barem u nekoj mjeri ne uključuje rad unutar tima. Također sve veći dio zaposlenika radi na izdvojenom radnom mjestu tj. \textit{remote} ili hibridno, a sve manji radi isključivo iz ureda tj. \textit{on-site}, pa je potrebno svladati i tu prepreku te nekadašnju komunikaciju isključivo licem u lice sada dijelom zamijeniti novim alatima. Kako bi tim uspješno djelovao i ostvario postavljeni cilj nužno je da su svi članovi tima međusobno usklađeni i mogu neometano surađivati - zato je potrebno osigurati da su članovi tima umreženi te mogu brzo primati novosti vezane uz poslovne zadatake dodijeljene timu te lako međusobno dijeliti resurse.

Kao praktični dio ovog završnog rada izrađena je web aplikacija \textit{Teamstructor}. Registrirani korisnici mogu kreirati timove i pozivati članove u svoj tim. Članovi tima mogu kreirati različite projekte unutar tima. Unutar svakog projekta članovi tima mogu raspravljati i objavljivati novosti relevantne za projekt u formi objava i jednorazinskih komentara. Također, unutar svakog tima članovi tima mogu pregledavati dijeljene resurse te prenositi nove datoteke s osobnog računala među resurse. Prijavljeni korisnici mogu pristupati timovima i timskim projektima (te raspravama i resursima) samo ukoliko su članovi tog tima. Implementirano je i jednostavno administrativno sučelje dostupno korisniku s ulogom administratora u kojemu može lako pregledavati postojeće korisnike, timove i projekte.

GitHub repozitorij u kojem je sadržan cijeli izvorni k\^od aplikacije \textit{Teamstructor} dosupan je na poveznici \url{https://github.com/anamarijapapic/teamstructor}, a repozitorij s \LaTeX \ datotekama korištenima za pisanje ovog rada na poveznici \url{https://github.com/anamarijapapic/teamstructor-docs}.

U poglavljima koja slijede prvo će kratko biti predstavljene tehnologije korištene pri izradi aplikacije, a zatim će detaljno i pregledno biti opisan način na koji je sama aplikacija implementirana.
